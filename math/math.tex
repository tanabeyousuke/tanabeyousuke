\documentclass[a4paper]{jsarticle}

\usepackage{amssymb, amsmath}
\usepackage{ascmac}
\usepackage{fancybox}
\usepackage{listings}

\title{'(レポート (ベクトル 計算) (対象 lisper))}
\author{渡邉 陽平}
\date{2024年4月1日}

\newcommand{\abs}[1]{\lvert #1 \rvert}
\newcommand{\norm}[1]{\lVert #1 \rVert}
\newcommand{\bekutoru}[1]{\overrightarrow{\rm #1}}
\newcommand{\itibekutoru}[2]{#1(\bekutoru{#2})}

\begin{document}

\maketitle

%% \section*{1 ベクトル}

%% 平面上で図形の平行移動は、向きを持つ線分で表現できる。
%% 線分につけた矢印の向きで移動する向きを、線分の長さで移動距離を表現する。

%% \subsection*{A 有効線分とベクトル}

%% 向きをつけた線分を有向線分という。
%% 有向線分ABでは、Aを始点、Bを終点と呼び、その向きはAからBへと向かっているとする。
%% また、線分ABの長さを、有向線分ABの大きさまたは長さと言う。

%% 平面上で図形を平行移動する場合,平行移動を表す線分はいくらでも図示できるが、位置が違うだけで長さ、向きは等しい。

%% 有向線分の違いを無視して、その向き、長さのみに着目したものをベクトルという。

%% 例として、物理で習った速度や力は向きと大きさを持つ量であり、ベクトルと言える。

%% \subsection*{B ベクトルの表記}
%% 有向線分ABが表すベクトルを$\bekutoru{AB}$ で表す。また、ベクトルを$\bekutoru{a}$ $\bekutoru{b}$ と表すこともある。
%% \par ベクトル$\bekutoru{AB}$ , $\bekutoru{a}$の大きさはそれぞれ$\abs{\bekutoru{AB}}$ , $\abs{\bekutoru{a}}$ で表す。このとき、$\abs{\bekutoru{AB}}$ は有向線分ABの長さに等しい。
%% \par 向き、大きさが同じの2つのベクトル$\bekutoru{a}$ ,$\bekutoru{b}$ は等しいといい、$\bekutoru{a}=\bekutoru{b}$ と書く。
%% \par$\bekutoru{AB}=\bekutoru{CD}$のとき、有向線分$\bekutoru{AB}$を平行移動して有向線分$\bekutoru{CD}$と重ね合わせることができる。

%% \par
%% ベクトル$\bekutoru{a}$と大きさが等しく、向きが反対のベクトルを$\bekutoru{a}$の逆ベクトルといい、$-\bekutoru{a}$で表す。

%% \begin{screen}
%%   $\bekutoru{a}=\bekutoru{BA}$である。\par
%%   すなわち$\bekutoru{BA}=\bekutoru{AB}$
%% \end{screen}

%% \section*{2 ベクトルの演算}
%% \subsection*{A ベクトルの加法}

%% ベクトル$\bekutoru{a}=\bekutoru{AB}$とベクトル$\bekutoru{b}$に対して、$\bekutoru{BC}=\bekutoru{b}$となる点Cを取る。\par
%% このようにして定まるベクトル$\bekutoru{AB}$を$\bekutoru{a}$と$\bekutoru{b}$の和といい、$\bekutoru{a}+\bekutoru{b}$と書く。\par
%% 次が成り立つ。
%% \begin{screen}
%%   $\bekutoru{AB}+\bekutoru{BC}=\bekutoru{AC}$
%% \end{screen}

%% ベクトルの加法について、次のことが成り立つ。
%% \begin{itembox}[l]{ベクトルの加法の性質}
%%   $\bekutoru{a}+\bekutoru{b}=\bekutoru{b}+\bekutoru{a}$ (交換法則)
%%   $(\bekutoru{a}+\bekutoru{b})+\bekutoru{c}=\bekutoru{a}+(\bekutoru{b}+\bekutoru{c})$ (結合法則)
%% \end{itembox}

%% \subsection*{B 零ベクトル}
%% $\bekutoru{a}=\bekutoru{AB}$のとき、$-\bekutoru{a}=\bekutoru{BA}$であるから、\par
%% $\bekutoru{a}+(-\bekutoru{a})=\bekutoru{AB}+\bekutoru{BA}=\bekutoru{AA}$\par となる。\par
%% ここで、$\bekutoru{AA}$は始点と終点が一致した有向線分のベクトルと考え、その大きさは0であるとする。\par
%% 大きさが0のベクトルを零ベクトルまたはゼロベクトルといい、$\bekutoru{0}$で表す。\par
%% 零ベクトルに関して、次が成り立つ。
%% \begin{screen}
%%   $\bekutoru{a}+(-\bekutoru{a})=\bekutoru{0}$\par
%%   $\bekutoru{a}+\bekutoru{0}=\bekutoru{a}$
%% \end{screen}

%% \subsection*{C ベクトルの減法}
%% ベクトル$\bekutoru{a}$, $\bekutoru{b}$に対して、$\bekutoru{b}+\bekutoru{c}=\bekutoru{a}$を満たすベクトル$\bekutoru{c}$を、$\bekutoru{a}$と$\bekutoru{b}$の差といい、$\bekutoru{a}-\bekutoru{b}$と書く。\par

%% 一般に、$\bekutoru{OB}+\bekutoru{BA}=\bekutoru{OA}$であるから、次が成り立つ。
%% \begin{screen}
%%   $\bekutoru{OA}-\bekutoru{OB}=\bekutoru{BA}$
%% \end{screen}

%% 同様に、$\bekutoru{OA}+\bekutoru{AB}=\bekutoru{OB}$より、次が成り立つ。\par
%% \begin{screen}
%%   $\bekutoru{AB}=\bekutoru{OB}-\bekutoru{OA}$
%% \end{screen}

%% ベクトルの減法について、次が成り立つ。
%% \begin{screen}
%%   $\bekutoru{a}-\bekutoru{b}=\bekutoru{a}+\bekutoru{b}$\par
%%   $\bekutoru{a}-\bekutoru{a}=\bekutoru{0}$
%% \end{screen}

\section*{ベクトルの成分}
\subsection*{ベクトルの成分表示}
x軸、y軸の正の向きと同じ向きの長さが1のベクトルを基本ベクトルといい、それぞれ$\bekutoru{e_1}, \bekutoru{e_2}$と表す。\par
ベクトルをxとyの増加量で書くと次のようになる。\par
$\bekutoru{a}=(a_1, a_2)$\par
$a_1, a_2$をそれぞれ$\bekutoru{a}$のx成分、y成分と言う。まとめて成分とも言う。\par
2つのベクトルについて次が成り立つ。\par
$\bekutoru{a}=\bekutoru{a}$であれば$a_1=b_1, a_2=b_2$
また、ベクトルの大きさは次のように求められる。\par
\begin{screen}
  $\bekutoru{a}=(a_1, a_2)$のとき、$\abs{\bekutoru{a}}=\sqrt{a^2_1+a^2_1}$
\end{screen}

\begin{boxnote}
  ベクトルの成分表示はcommon lispのconsセルを用いて以下のように表せる。\par
  \begin{lstlisting}
    (x . y)
  \end{lstlisting}

  ベクトルの大きさを求める関数をcommon lispで書くと、\par
  \begin{lstlisting}
    (defun size (vector)
      (sqrt (+
        (expt (car vector) 2)
        (expt (cdr vector) 2))))
  \end{lstlisting}
\end{boxnote}

\subsection*{和,差,実数倍の成分表示}

ベクトルの和、差、実数倍は次のことが言える。
\begin{screen}
  $(a_1, a_2)+(b_1, b_2)=(a_1+b_1, a_2+b_2)$\par
  $(a_1, a_2)-(b_1, b_2)=(a_1-b_1, a_2-b_2)$\par
  $k(a_1, a_2)=(ka_1, ka_2)$\par
\end{screen}
要はそれぞれの成分同士を足したり引いたり、それぞれの成分にかけてやれば良い。

\begin{boxnote}
  ベクトルの和、差、実数倍を求める関数は
  \begin{lstlisting}
    (defun wa (firstvector secondvector)
      (cons
        (+ (car firstvector) (car secondvector))
        (+ (cdr firstvector) (cdr secondvector))))

    (defun sa (firstvector secondvector)
      (cons
        (- (car firstvector) (car secondvector))
        (- (cdr firstvector) (cdr secondvector))))

    (defun zissuubai (vector number)
      (cons
        (* (car vector) number)
        (* (cdr vector) number)))
    
  \end{lstlisting}  
\end{boxnote}

\subsection*{座標平面上の点とベクトル}

座標平面上に2点$A(a_1, a_2),B(b_1, b_2)$がある時,$\bekutoru{AB}$は
$\bekutoru{AB}=\bekutoru{OB}-\bekutoru{OA}$\par
$(a_1, a_2)-(b_1, b_2)=(a_1-b_1, a_2-b_2)$\par

以上から次が言える。
\begin{screen}
  二点$A(a_1, a_2),B(b_1, b_2)$について
  $\bekutoru{AB}=(a_1-b_1, a_2-b_2)$\par
  $\abs{\bekutoru{AB}}=\sqrt{(a_1-b_1, a_2-b_2)}$\par
\end{screen}

上は差を求める関数を使えばよいのでlispのコードは書かない。\par

\section*{ベクトルの内積}
\subsection*{ベクトルの内積}

大きさが0ではない2つのベクトル$\bekutoru{a},\bekutoru{b}$について、適当な点Oから伸ばした先をA,Bとする。
$\angle\mathrm{AOB}$の優角でない方を$\bekutoru{a},\bekutoru{b}$のなす角という。\par
$\abs{\bekutoru{a}}\abs{\bekutoru{b}}\cos\theta$を内積という。$\bekutoru{a}\cdot\bekutoru{b}$と書く。

\subsection*{成分による内積の表示}
ベクトルの内積は成分表示を用いて次のように表すことができる。

\begin{screen}
  $\bekutoru{a}\cdot\bekutoru{b}=a_1b_1+a_2b_2$
\end{screen}

\begin{boxnote}
  内積を求める関数は
  \begin{lstlisting}
  (defun naiseki (firstvector secondvector)
    (+ (* (car firstvector) (car secondvector))
       (* (cdr firstvector) (cdr secondvector))))
  \end{lstlisting}
  なす角を求める関数は
  \begin{lstlisting}
  (defun nasukaku (firstvector secondvector)
    (acos (/
      (naiseki firstvector secondvector)
        (* (size firstvector) (size secondvector)))))
  \end{lstlisting}
\end{boxnote}

なす角が$90^{\circ}$のときは内積は0になる。$\cos90^{\circ}$は0だからである。仕方ないね。

\subsection*{内積の性質}
次が成り立つ。
\begin{screen}
  $\bekutoru{a}\cdot\bekutoru{a}=\abs{\bekutoru{a}}^2$\par
  $\bekutoru{a}\cdot\bekutoru{b}=\bekutoru{b}\cdot\bekutoru{a}$\par
  $(\bekutoru{a}+\bekutoru{b})\cdot\bekutoru{c}=\bekutoru{a}\cdot\bekutoru{c}+\bekutoru{b}\cdot\bekutoru{c}$\par
  $\bekutoru{a}\cdot(\bekutoru{b}+\bekutoru{c})=\bekutoru{a}\cdot\bekutoru{c}+\bekutoru{b}\cdot\bekutoru{c}$\par
  $(k\bekutoru{a})\cdot\bekutoru{b}=\bekutoru{a}\cdot(k\bekutoru{b})=k(\bekutoru{a}\cdot\bekutoru{b})$\par
\end{screen}



\section*{位置ベクトル}
\subsection*{位置ベクトル}
平面上で点Oを決めておくとどんな点Pでもベクトル$\bekutoru{p}=\bekutoru{OP}$によって決まる。\par
これを位置ベクトルと呼び、上の例では$\itibekutoru{P}{p}$で表す。\par

二点A,Bに対して、
$\bekutoru{AB}=\bekutoru{b}-\bekutoru{a}$が言える。

\subsection*{内分点、外分点の位置ベクトル}

2点$\itibekutoru{A}{a},\itibekutoru{B}{b}$にたいして、線分ABをm:nに内分,外分する点の位置ベクトルは次のようになる。
\begin{screen}
  内分:\par
  $\frac{n\bekutoru{a}+m\bekutoru{b}}{m+n}$\par
  外分:\par
  $\frac{-n\bekutoru{a}+m\bekutoru{b}}{m-n}$
\end{screen}

\begin{boxnote}
  内分、外分する位置ベクトルを求める関数は
  \begin{lstlisting}
    (defun naibun (firstvector secondvector m n)
      (/
        (+ (zissubai firstvector n) (zissubai secondvector m))
        (+ m n)))

    (defun naibun (firstvector secondvector m n)
      (/
        (+ (- 0 (zissubai firstvector n)) (zissubai secondvector m))
        (- m n)))
  \end{lstlisting}
\end{boxnote}

\subsection*{三角形の重心の位置ベクトル}

三角形の重心の位置ベクトル$\bekutoru{g}$を求める。
位置ベクトル$\bekutoru{g}$は
$\bekutoru{g}=\frac{\bekutoru{a}+\bekutoru{b}+\bekutoru{c}}{3}$

\begin{boxnote}
  重心を求める関数は
  \begin{lstlisting}
    (defun zyuusin (firstvector secondvector thirdvector)
      (/ (wa firstvector (wa secondvector thirdvector)) 3))
  \end{lstlisting}
\end{boxnote}

\end{document}
