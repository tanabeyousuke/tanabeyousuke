\documentclass[pulatex,dvpdfmx,a4paper]{jsarticle}

\usepackage{amssymb, amsmath}

\title{ベクトル}
\author{渡邉 陽平}
\date{2024年4月1日}


\newcommand{\abs}[1]{\lvert #1 \rvert}
\newcommand{\norm}[1]{\lVert #1 \rVert}

\begin{document}

\maketitle

\section*{1 ベクトル}

平面上で図形の平行移動は、向きを持つ線分で表現できる。
線分につけた矢印の向きで移動する向きを、線分の長さで移動距離を表現する。

\subsection*{A 有効線分とベクトル}

向きをつけた線分を有向線分という。
有向線分ABでは、Aを始点、Bを終点と呼び、その向きはAからBへと向かっているとする。
また、線分ABの長さを、有向線分ABの大きさまたは長さと言う。

平面上で図形を平行移動する場合,平行移動を表す線分はいくらでも図示できるが、位置が違うだけで長さ、向きは等しい。

有向線分の違いを無視して、その向き、長さのみに着目したものをベクトルという。

例として、物理で習った速度や力は向きと大きさを持つ量であり、ベクトルと言える。

\subsection*{B ベクトルの表記}
有向線分ABが表すベクトルを\overrightarrow{\rm AB} で表す。また、ベクトルを\overrightarrow{\rm a}, \overrightarrow{\rm b} と表すこともある。
\par ベクトル\overrightarrow{\rm AB} , \overrightarrow{\rm a} の大きさはそれぞれ\abs{\overrightarrow{\rm AB}} , \abs{\overrightarrow{\rm a}} で表す。このとき、\abs{\overrightarrow{\rm AB}} は有向線分ABの長さに等しい。
\par 向き、大きさが同じの2つのベクトル\overrightarrow{\rm a} ,\overrightarrow{\rm b} は等しいといい、\overrightarrow{\rm a} =\overrightarrow{\rm b} と書く。
\par\overrightarrow{\rm AB}=\overrightarrow{\rm CD}のとき、有向線分\overrightarrow{\rm AB}を平行移動して有向線分\overrightarrow{\rm CD}に重ね合わせることができる。

\par
ベクトル\overrightarrow{\rm a}と大きさが等しく、向きが反対のベクトルを\overrightarrow{\rm a}の逆ベクトルといい、-\overrightarrow{\rm a}で表す。
\end{document}
