\documentclass[pulatex,dvpdfmx,a4paper]{jsarticle}

\usepackage{amssymb, amsmath}
\usepackage{ascmac}
\usepackage{fancybox}

\title{ベクトル}
\author{渡邉 陽平}
\date{2024年4月1日}


\newcommand{\abs}[1]{\lvert #1 \rvert}
\newcommand{\norm}[1]{\lVert #1 \rVert}
\newcommand{\bekutoru}[1]{\overrightarrow{\rm #1}}

\begin{document}

\maketitle

\section*{1 ベクトル}

平面上で図形の平行移動は、向きを持つ線分で表現できる。
線分につけた矢印の向きで移動する向きを、線分の長さで移動距離を表現する。

\subsection*{A 有効線分とベクトル}

向きをつけた線分を有向線分という。
有向線分ABでは、Aを始点、Bを終点と呼び、その向きはAからBへと向かっているとする。
また、線分ABの長さを、有向線分ABの大きさまたは長さと言う。

平面上で図形を平行移動する場合,平行移動を表す線分はいくらでも図示できるが、位置が違うだけで長さ、向きは等しい。

有向線分の違いを無視して、その向き、長さのみに着目したものをベクトルという。

例として、物理で習った速度や力は向きと大きさを持つ量であり、ベクトルと言える。

\subsection*{B ベクトルの表記}
有向線分ABが表すベクトルを$\bekutoru{AB}$ で表す。また、ベクトルを$\bekutoru{a}$ $\bekutoru{b}$ と表すこともある。
\par ベクトル$\bekutoru{AB}$ , $\bekutoru{a}$の大きさはそれぞれ$\abs{\bekutoru{AB}}$ , $\abs{\bekutoru{a}}$ で表す。このとき、$\abs{\bekutoru{AB}}$ は有向線分ABの長さに等しい。
\par 向き、大きさが同じの2つのベクトル$\bekutoru{a}$ ,$\bekutoru{b}$ は等しいといい、$\bekutoru{a}=\bekutoru{b}$ と書く。
\par$\bekutoru{AB}=\bekutoru{CD}$のとき、有向線分$\bekutoru{AB}$を平行移動して有向線分$\bekutoru{CD}$と重ね合わせることができる。

\par
ベクトル$\bekutoru{a}$と大きさが等しく、向きが反対のベクトルを$\bekutoru{a}$の逆ベクトルといい、$-\bekutoru{a}$で表す。

\begin{screen}
  $\bekutoru{a}=\bekutoru{BA}$である。\par
  すなわち$\bekutoru{BA}=\bekutoru{AB}$
\end{screen}

\section*{2 ベクトルの演算}
\subsection*{A ベクトルの加法}

ベクトル$\bekutoru{a}=\bekutoru{AB}$とベクトル$\bekutoru{b}$に対して、$\bekutoru{BC}=\bekutoru{b}$となる点Cを取る。\par
このようにして定まるベクトル$\bekutoru{AB}$を$\bekutoru{a}$と$\bekutoru{b}$の和といい、$\bekutoru{a}+\bekutoru{b}$と書く。\par
次が成り立つ。
\begin{screen}
  $\bekutoru{AB}+\bekutoru{BC}=\bekutoru{AC}$
\end{screen}

ベクトルの加法について、次のことが成り立つ。
\begin{itembox}[l]{ベクトルの加法の性質}
  $\bekutoru{a}+\bekutoru{b}=\bekutoru{b}+\bekutoru{a}$ (交換法則)
  $(\bekutoru{a}+\bekutoru{b})+\bekutoru{c}=\bekutoru{a}+(\bekutoru{b}+\bekutoru{c})$ (結合法則)
\end{itembox}

\subsection*{B 零ベクトル}
$\bekutoru{a}=\bekutoru{AB}$のとき、$-\bekutoru{a}=\bekutoru{BA}$であるから、\par
$\bekutoru{a}+(-\bekutoru{a})=\bekutoru{AB}+\bekutoru{BA}=\bekutoru{AA}$\par となる。\par
ここで、$\bekutoru{AA}$は始点と終点が一致した有向線分のベクトルと考え、その大きさは0であるとする。\par
大きさが0のベクトルを零ベクトルまたはゼロベクトルといい、$\bekutoru{0}$で表す。\par
零ベクトルに関して、次が成り立つ。
\begin{screen}
  $\bekutoru{a}+(-\bekutoru{a})=\bekutoru{0}$\par
  $\bekutoru{a}+\bekutoru{0}=\bekutoru{a}$
\end{screen}

\subsection*{C ベクトルの減法}
ベクトル$\bekutoru{a}$, $\bekutoru{b}$に対して、$\bekutoru{b}+\bekutoru{c}=\bekutoru{a}$を満たすベクトル$\bekutoru{c}$を、$\bekutoru{a}$と$\bekutoru{b}$の差といい、$\bekutoru{a}-\bekutoru{b}$と書く。\par

一般に、$\bekutoru{OB}+\bekutoru{BA}=\bekutoru{OA}$であるから、次が成り立つ。
\begin{screen}
  $\bekutoru{OA}-\bekutoru{OB}=\bekutoru{BA}$
\end{screen}

同様に、$\bekutoru{OA}+\bekutoru{AB}=\bekutoru{OB}$より、次が成り立つ。\par
\begin{screen}
  $\bekutoru{AB}=\bekutoru{OB}-\bekutoru{OA}$
\end{screen}

ベクトルの減法について、次が成り立つ。
\begin{screen}
  $\bekutoru{a}-\bekutoru{b}=\bekutoru{a}+\bekutoru{b}$\par
  $\bekutoru{a}-\bekutoru{a}=\bekutoru{0}$
\end{screen}


\end{document}
